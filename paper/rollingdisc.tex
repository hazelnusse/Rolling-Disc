\documentclass[letterpaper,11pt]{article}
%\usepackage[round]{natbib}
\usepackage[margin=1in,centering]{geometry}
\usepackage{fancyhdr}
\usepackage{amsmath}
\usepackage{amssymb}
\usepackage{pdftricks}
\begin{psinputs}
  \usepackage{pstricks}
  \usepackage{pst-vue3d}
\end{psinputs}
%\usepackage{graphicx}
\usepackage[pdftex]{hyperref}
\hypersetup{
    pdftitle={The Rolling Disc},
    pdfauthor={Dale Lukas Peterson},
    pdfsubject={Nonholonomic mechanics},
    pdfkeywords={Dynamics}}

\renewcommand{\b}[1]{ \mathbf{ #1 } }
\newcommand{\bs}[1]{\boldsymbol{#1}}

\pagestyle{fancy}
\fancyhead[L]{The Rolling Disc, Dale L. Peterson}
\fancyhead[R]{\thepage}  % page number on the right
\fancyfoot[L,C,R]{}  %  No footer on left, center or right, on even or odd pages

\begin{document}

\section{Model description}

Consider a thin knife edged disc $C$ of radius $r$, mass $m$, and inertia about
it's mass center of $\mathbf{I}^{C/CO} = I \bs{c}_x\bs{c}_x + J
\bs{c}_y\bs{c}_y + I \bs{c}_z\bs{c}_z$.  The body fixed reference frame $C$ has
the $\bs{c}_x$ and $\bs{c}_z$ unit vectors in the plane of the disc and
$\bs{c}_y$ perpendicular to the plane of the disc.  We denote the center of
disc as $CO$.

Consider an inertial reference frame $N$ with $\bs{n}_z$ aligned with the local
gravitational field (``down''), and $\bs{n}_x$ and $\bs{n}_y$ spanning the ``ground
plane''.  We denote the origin of this inertial frame as $NO$.

We use five generalized coordinates to completely configure $C$ in $N$.  Two
coordinates $x$ and $y$ locate the contact point of the disc with the ground
plane.  Euler Z-X-Y ($\psi-\phi-\theta$) angles are used to orient $C$ in
$N$.  $\psi$ measures the heading angle (yaw) of the path of the disc relative
$\bs{n}_x$, $\phi$ measures the lean angle (roll) of the plane of the disc
relative to a vertical plane, and $\theta$ measures the spin angle of the disc
about it's axis of symmetry.

% We shall see that the dynamic equations depend upon only one of the generalized coordinates, namely the lean angle $\phi$.

We introduce two intermediate reference frames, $A$ and $B$, the ``yaw'' and
``roll'' frames, respectively.  These are the intermediate frames
obtained during the sequence of Euler rotations.  As a consequence $\bs{n}_z =
\bs{a}_z$, $\bs{a}_x = \bs{b}_x$, and $\bs{b}_y = \bs{c}_y$.

The disc is assumed to make point contact with the ground plane and to roll
without slipping.  For any instant in time, there are four points which are
coincident to the contact point:
\begin{itemize}
  \item[$NC$] Fixed in the inertial frame, instantaneously touches the disc as
    it rolls past.
  \item[$\overline{NC}$] Moving in the ground plane frame, constantly touches
    the disc as it rolls.
  \item[$CN$] Fixed on the perimeter of the disc, instantaneously touches the
    ground.
  \item[$\overline{CN}$] Moving on the disc perimeter, constantly touches the
    ground as the disc rolls.
\end{itemize}
The velocity of points $NC$ and $CN$, relative to the inertial frame,
are zero because $NC$ is fixed in the inertial plane (hence has zero velocity)
and because the disc is assumed to roll without slipping.  The velocities of
$\overline{NC}$ and $\overline{CN}$, however, will not necessarily be zero, but
they will be equal to each other.  The distinction here is subtle because all
of these points are coincident (same position), yet they all do not have the
same velocity.

%We denote the point fixed to the disc which is instantaneously
%in contact with the ground as $CN$.  The point coincident with this point, but
%moving relative to inertial frame we denote as $NC$.

%The yaw frame $A$ is obtained by first aligning $A$
%with $N$ and then performing a simple rotation of $A$ about $\bs{n}_z$ by an
%angle $\psi$.  The lean frame $B$ is obtained by first  aligning $B$
%with $A$ and then performing a simple rotation of $B$ about $\bs{a}_x$ by an
%angle $\phi$.  This then implies that $\bs{n}_z = \bs{a}_z$ and $\bs{a}_x =
%\bs{b}_x$.

\section{Equations of motion}

\subsection{Kinematics}

\subsubsection{Angular velocities, velocities, and no-slip constraints}

Due to the symmetry of the disc about $\bs{b}_y$, the inertia of the disc can
be expressed in the coordinates of the $B$ frame, rather than the
body fixed frame $C$:  $\bs{I}^{C/CO} = I \bs{b}_x\bs{b}_x + J \bs{b}_y\bs{b}_y
+ I \bs{b}_z\bs{b}_z$.  For this reason, we define three generalized speeds
\begin{equation}
  \begin{tabular}{ccccccc}
    $\omega_x \triangleq {}^{N}\bs{\omega}^C \cdot \bs{b}_x$ & & &
    $\omega_y \triangleq {}^{N}\bs{\omega}^C \cdot \bs{b}_y$ & & &
    $\omega_z \triangleq {}^{N}\bs{\omega}^C \cdot \bs{b}_z$
  \end{tabular}
  \label{eq:genspeeds}
\end{equation}

We can also express ${}^N\bs{\omega}^C$ in terms of the time derivatives of the Euler angles as
\begin{align}
  {}^N\bs{\omega}^C & = \dot{\psi} \bs{a}_z + \dot{\phi} \bs{b}_x +
  \dot{\theta} \bs{b}_y
  \label{eq:omega}
\end{align}

The relationship between the generalized speeds defined in \ref{eq:genspeeds}
and the time derivatives of the Euler angles can be determined by carrying out
the dot products in
(\ref{eq:genspeeds}) while making use of (\ref{eq:omega}).  This results in
\begin{equation}
  \begin{tabular}{ccccccc}
    $\omega_x = \dot{\phi}$ & & &
    $\omega_y = \dot{\theta} + \dot{\psi}\sin{\phi}$ & & &
    $\omega_z = \dot{\psi} \cos{\phi}$
  \end{tabular}
  \label{eq:kindiffs1}
\end{equation}

The velocity of the disc center ${}^N\bs{v}^{CO}$ must be formed regardless of
which method is to be employed to derived the dynamic equations of motion
(Kane's, Lagrange's, Hamilton's or Newton-Euler).  The velocity
of the mass center can be formed in two ways, thus providing a natural way by
which to derive the no-slip conditions (since the two ways must give the same
velocity).

The first way is to recognize that to prevent slip, ${}^N\bs{v}^{CN} = 0$.
Since $CN$ and $CO$ are both fixed to the disc, we have
\begin{align}
    {}^N\bs{v}^{CO} &= {}^N\bs{v}^{CN} + {}^N\bs{\omega}^C \times
    (\bs{r}^{CO/CN}) \nonumber \\
    &= \bs{0} + (\dot{\psi} \bs{a}_z + \dot{\phi} \bs{b}_x
    + \dot{\theta} \bs{b}_y) \times (-r\bs{b}_z) \nonumber \\
  &= -r(\dot{\theta} + \dot{\psi}\sin{\phi})\bs{b}_x
     + r\dot{\phi}\bs{b}_y 
  \label{eq:vco1}
\end{align}

The second way is to consider a rod of length $r$ with one end  (moving in $N$)
and the other end attached to the disc mass center.  The rod is fixed in the
$B$ reference frame, therefore
\begin{align}
    {}^N\bs{v}^{CO} &= \frac{ {}^Nd}{dt}(\bs{r}^{\overline{NC}/NO}
    + \bs{r}^{CO/\overline{NC}}) \nonumber \\
    &= \frac{ {}^Nd}{dt}(x\bs{n}_x + y\bs{n}_y - r \bs{b}_z) \nonumber \\
%    &=  \dot{x} \bs{n}_x + \dot{y} \bs{n}_y 
%    - {}^N\bs{\omega}^B \times r \bs{b}_z \nonumber \\
                  &=  \dot{x} \bs{n}_x + \dot{y} \bs{n}_y - r\dot{\psi}\sin{\phi}\bs{b}_x + r\dot{\phi}\bs{b}_y
  \label{eq:vco2}
\end{align}

By equating (\ref{eq:vco1}) with (\ref{eq:vco2}) and rearranging, we obtain
\begin{equation}
  r\dot{\theta}\bs{b}_x + \dot{x} \bs{n}_x + \dot{y} \bs{n}_y = \bs{0}
  \label{eq:nhc_vec}
\end{equation}
Equation (\ref{eq:nhc_vec}) is the vector form of the no slip rolling
condition.  Resolving (\ref{eq:nhc_vec}) into components of the $N$ frame, we
obtain
\begin{equation}
  \begin{tabular}{ccccccc}
  $\dot{x} + r\dot{\theta}\cos{\psi} = 0$ & & & & &
  $\dot{y} + r\dot{\theta}\sin{\psi} = 0$
  \end{tabular}
  \label{eq:nhc_scalar}
\end{equation}
which are the scalar form of the no slip rolling condition.

It is important to realize that given $\omega_x$, $\omega_y$, and $\omega_z$,
the first time derivative of \textit{all} generalized coordinates are
completely determined by \ref{eq:kindiffs1} and \ref{eq:nhc_scalar}.  The
generalized speeds change according to a dynamic differential equation.

At this point, if one uses Lagrange's or Hamilton's methods to form the
equations of motion, the kinematic analysis ends since the kinetic and
potential energies can be formed, and the constraints have been written in a
form suitable for employing Lagrange multipliers.  Using Kane's method or
the Newton-Euler method, accelerations (angular and translational) must be
formed.  We proceed with these subsequently.

\subsubsection{Angular accelerations, accelerations}
Rather than utilizing the angular velocity and velocity expressions which
explicitly depend upon the first time derivatives of the generalized
coordinates, we employ expressions which depend upon the generalized speeds.
This approach directly yields dynamic equations in first order form.

The angular acceleration is
\begin{align}
    {}^N\bs{\alpha}^{C} &= \frac{ {}^Nd}{dt}{}^N\bs{\omega}^C \nonumber \\
    &= \frac{ {}^Bd}{dt} {}^N\bs{\omega}^C + {}^N\bs{\omega}^B \times
    {}^N\bs{\omega}^C
    \nonumber \\
    &= \frac{ {}^Bd}{dt}(\omega_x \bs{b}_x + \omega_y \bs{b}_y +\omega_z
    \bs{b}_z) + (\omega_x \bs{b}_x + \omega_z \tan{\phi} \bs{b}_y + \omega_z
    \bs{b}_z) \times (\omega_x \bs{b}_x + \omega_y \bs{b}_y +\omega_z
    \bs{b}_z) \nonumber \\
    & = (\omega_z^2\tan{\phi} + \dot{\omega}_x- \omega_y\omega_z)\bs{b}_x 
    + \dot{\omega}_y\bs{b}_y + (\omega_x\omega_y +\dot{\omega}_z -
    \omega_x\omega_z\tan{\phi}) \bs{b}_z
  \label{eq:alf}
\end{align}

The acceleration of the disc mass center is
\begin{align}
    {}^N\bs{a}^{CO} &= \frac{ {}^Nd}{dt}{}^N\bs{v}^{CO} \nonumber \\
    &= \frac{ {}^Bd}{dt} {}^N\bs{v}^{CO} + {}^N\bs{\omega}^B \times
    {}^N\bs{v}^{CO}
    \nonumber \\
    &= \frac{ {}^Bd}{dt} (-r \omega_y \bs{b}_x + r \omega_x \bs{b}_y) + (\omega_x \bs{b}_x + \omega_z \tan{\phi} \bs{b}_y + \omega_z
    \bs{b}_z) \times (-r \omega_y \bs{b}_x + r \omega_x \bs{b}_y) \nonumber
    \\
    &= -r (\omega_x\omega_z + \dot{\omega}_y)\bs{b}_x + r (\dot{\omega}_x -
    \omega_y \omega_z)\bs{b}_y + r (\omega_x^2 + \omega_y
    \omega_z\tan{\phi})\bs{b}_z
  \label{eq:aco}
\end{align}


\subsection{Dynamics}
There are two forces acting on the disc: gravity at the mass center and
reaction forces at the ground contact point.  The gravitational force
acting at $CO$ is
\begin{align}
    \bs{F}^{CO} &= mg\bs{a}_z
    \label{eq:fco}
\end{align}
while the contact forces are
\begin{align}
    \bs{F}^{CN} &= F_x \bs{a}_x + F_y \bs{a}_y + F_z \bs{a}_z
    \label{eq:fcn}
\end{align}
where the contact force measure numbers have been taken in the $A$ reference
frame (the yaw frame).  

We now proceed to derive the equations of motion using Kane's method,
Newton-Euler, Lagranges, and finally Hamilton's method.

\subsubsection{Kane's Method}

The most distinct feature of Kane's method are the partial velocities and
partial angular velocities.  These are nothing more than the partial
derivatives of the velocity and angular velocity expressions with respect to
each of the generalized speeds.  Velocity expressions will always be linear in
the generalized speeds (which are in turn linear in the first time derivative
of the coordinates), therefore the partial velocities may be obtained simply by
determining the vector coefficient of each generalized speed.  We proceed now
under the assumption that only the motion equations are desired and forces of
constraint are not of interest.  We will return to the constraint forces
subsequently.

The partial angular velocities of $C$ are
\begin{equation}
  \begin{tabular}{ccccccc}
      $\bs{\omega}^C_x = \bs{b}_x$ & & &
      $\bs{\omega}^C_y = \bs{b}_y$ & & &
      $\bs{\omega}^C_z = \bs{b}_z$
  \end{tabular}
  \label{eq:pav}
\end{equation}

The partial velocities of $CO$ are
\begin{equation}
  \begin{tabular}{ccccccc}
      $\bs{v}^{CO}_x = r \bs{b}_y$ & & &
      $\bs{v}^{CO}_y = -r \bs{b}_x$ & & &
      $\bs{v}^{CO}_z = \bs{0}$
  \end{tabular}
  \label{eq:pv}
\end{equation}

Because the three generalized speeds are all independent, we can directly
proceed to form the motion equations.

Kane's equations can be written as
\begin{align}
    (\bs{F}^{CO} - m {}^N\bs{a}^{CO})\cdot \bs{v}^{CO}_i
    + (-\bs{I}^{C/CO} \cdot {}^N\bs{\alpha}^C
       - {}^N\bs{\omega}^C \times \bs{I}^{C/CO} \cdot {}^N\bs{\omega}^C)
       \cdot \bs{\omega}^C_i & = 0 \qquad (i = x, y, z)
       \label{eq:kane}
\end{align}

Upon inspection, a few things might appear startling to those not familiar with
Kane's method.  The first term in (\ref{eq:kane}) is essentially Newton's
equation for the mass center, except the constraint forces are not included in
the resultant.  In the second term, a glimmer of Euler's equation is apparent,
we see the time rate of change of angular momentum, except there is no torque term due to the constraint forces.  Indeed, Kane's
equations can be viewed as containing at their core the Newton-Euler equations.
The beauty of Kane's method is that work-less constraint forces (torques), when
dotted with the partial velocities (partial angular velocities), are zero.  As
a result, they can be left out entirely, and algebraic elimination of
constraint forces is unnecessary.  While not an arduous task in a single body
system, it becomes quickly overwhelming in a multibody system.  Computing the
dot products in (\ref{eq:kane}) gives three equations of motion
\begin{align}
    % First EOM
    mgr\sin{\phi} + (mr^2 + J)\omega_y\omega_z -
    I\omega_z^2\tan{\phi} - (mr^2+I)\dot{\omega_x} &= 0 \\
    % Second EOM
       -mr^2\omega_x\omega_z - (mr^2 + J)\dot{\omega}_y &= 0 \\
    % Third EOM
    I\omega_x\omega_z\tan{\phi} - J\omega_x\omega_y - I\dot{\omega}_z &= 0
    \label{eq:eoms}
\end{align}

There are several important features of these equations:
\begin{itemize}

    \item The mass matrix is diagonal with respect to $\dot{\omega}_x$,
        $\dot{\omega}_y$, $\dot{\omega}_z$, and the equations are in first
        order form, ideal for simulation or for stability analysis.  Instead of
        second time derivatives of generalized coordinates, we instead have
        first tiem derivatives of generalized speeds.

    \item The constraint force terms do not appear in the equations and no
        algebraic elimination of them is needed.

    \item The only generalized coordinate that appears in the dynamic
        equations is $\phi$. This is a direct proof that $x$, $y$,
        $\psi$, and $\theta$ are cyclic (ignorable) as far as the dynamics are
        concerned.  In simpler terms, the acceleration of the disc are
        influenced only by the lean angle $\phi$, no where it is in the ground
        plane or the direction it is rolling.
\end{itemize}


\subsubsection{Lagrange's Method}

The Lagrangian is computed from the kinetic and potential energies.  The
kinetic energy can be expressed with time derivatives of the coordinates
($\dot{q}$'s or with the generalized speeds ($\omega$'s).  In the latter case,
one would need to employ the chain rule to apply Lagranges equations.  While
either approch is mathematically equivalent, we chose the former.  The kinetic
energy is
\begin{align}
    T &=  \frac{1}{2}m ({}^N\bs{v}^{CO} \cdot {}^N\bs{v}^{CO} + {}^N\bs{\omega}^C
    \cdot \bs{I}^{C/CO} \cdot {}^N\bs{\omega}^C) \nonumber \\
    &= \frac{1}{2}\left( (r^2 + I) \dot{\phi}^2 + I\dot{\psi}^2\cos^2{\phi}
      + (r^2 + J)(\dot{\theta}+\dot{\psi}\sin{\phi})^2 \right)
    \label{eq:ke}
\end{align}
while the potential energy can be defined as
\begin{align}
    V &= \bs{r}^{CO/CN} \cdot mg\bs{a}_z\nonumber \\
      &= -mgr\cos{\phi}
    \label{eq:pe}
\end{align}
In 


We now write the complete set of eight first order ODE's (five kinematic, three
dynamic) which completely govern the motion of the rolling disc.  They are

\begin{align}
    \label{eq:psidot}
    \dot{\psi} &= \omega_z \sec{\phi} \\
    \label{eq:phidot}
    \dot{\phi} &= \omega_x \\
    \label{eq:thetadot}
    \dot{\theta} &= \omega_y - \omega_z \tan{\phi} \\
    \label{eq:xdot}
    \dot{x} &= -r\dot{\theta}\cos{\psi} = r(\omega_z \tan{\phi} - \omega_y)\cos{\psi} \\
    \label{eq:ydot}
    \dot{y} &= -r\dot{\theta}\sin{\psi} = r(\omega_z \tan{\phi} - \omega_y)\sin{\psi} \\
    \label{eq:omegaxdot}
    \dot{\omega}_x &= (mgr\sin{\phi} + (mr^2 + J)\omega_y\omega_z -
    I\omega_z^2\tan{\phi})/(mr^2 + I) \\
    \label{eq:omegaydot}
    \dot{\omega}_y &= mr^2\omega_x\omega_z/(mr^2 + J) \\
    \label{eq:omegazdot}
    \dot{\omega}_z &= \omega_x(\omega_z \tan{\phi} - \omega_y J/I)
\end{align}

Equations (\ref{eq:psidot})-(\ref{eq:omegazdot}) permit two types of
equilibrium: static and dynamic.  Static equilibrium occurs when the right hand
sides of (\ref{eq:psidot})-(\ref{eq:omegazdot}) zero, which implies
$\omega_x=\omega_y=\omega_z=0$ and $\phi=\pm\pi/2$.  Dynamic equilibrium is
less restrictive requiring that the right hand sides of only 
(\ref{eq:phidot}), (\ref{eq:omegaxdot}), (\ref{eq:omegaydot}), and
(\ref{eq:omegazdot}) be zero.  For dynamic equilibrium, this implies that
\begin{align}
    mgr\sin{\phi} + (J + mr^2)\omega_y\omega_z - I\omega_z^2\tan{\phi} & = 0
    \label{eq:dyneq}
\end{align}



\end{document}
